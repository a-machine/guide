\documentclass{article}

\usepackage{hyperref}
\begin{document}

\title{Welcome to ADA!}

\maketitle


This is a quickstart guide for you to learn about how to utilize the ADA-pipeline to publishing a book.

\begin{quote}



\href{https://write.handbuch.io/}{Logon} here to make your book.


\end{quote}


The \href{https://github.com/TIBHannover/ADA}{ADA-Pipeline} joins together open-source platforms for book publishing.


\subsection{What you will learn here:}\label{H7757657}


\begin{quote}



ADA supports realtime collaborative authoring with an online word processor and publishing as multi-format using Git. You can distribute your book as: A website, paged website, PDF, print-on-demand, eBook, and more.


\end{quote}

\begin{enumerate}
\item How to prepare your public Git repository for storing your book data.


\item To setup your book's online collaborative word processor.


\item Invite your team to collaborate on writing online.


\item Publishing your book. 


\end{enumerate}

Any questions or comments looks us up on Github - \href{https://github.com/TIBHannover/ADA/discussions}{discussion} or raise and \href{https://github.com/orgs/TIBHannover/projects/2}{issue}.


Brought to you by team \href{https://projects.tib.eu/nextgen-books/en/}{\#NextGenBooks} service of the Open Science Lab, TIB. \href{https://creativecommons.org/licenses/by-sa/4.0/}{CC BY-SA 4.0}, 2022. Imprint and data pivacy.


email: Simon Worthington, \href{mailto:simon.worthington@tib.eu}{simon.worthington@tib.eu}


\href{https://projects.tib.eu/nextgen-books/en/contact/}{Contact} | \href{https://projects.tib.eu/nextgen-books/en/data-protection/}{Data protection} | \href{https://projects.tib.eu/nextgen-books/en/accessibility-statement/}{Accessibility Statement} | \href{https://projects.tib.eu/nextgen-books/en/imprint/}{Imprint}

\end{document}
